\documentclass[]{article}
    \usepackage{hyperref}
    \usepackage{longtable}
    \usepackage{csquotes}
    \usepackage{adjustbox}
    \usepackage{multirow}
    \usepackage{pdflscape}
    \usepackage[margin=0.5in]{geometry}
    \usepackage{todonotes}
    \usepackage{multirow}
    \usepackage{booktabs}


    %opening
    \title{Evaluation Appendix}
    \author{Simina Ana Cotop}
    \date{}
    \frenchspacing
    \begin{document}
    \begin{landscape}
    \maketitle
    \section{OptA}

	\begin{longtable}{|p{13mm}|p{57mm}|p{57mm}|p{57mm}|p{57mm}|}
	\hline 
	Chart & Human Description & Test  \\ 
	\hline 
	\endhead\multirow{2}{*}{Chart 0}&Women are more likely to represents law firms and financial groups than insurance and tech by nearly double 
 & Women are more represented in law firms than financial groups. Insurance is double that.
\newline \textbf{BLEU Score: }0.0\\ \cline{2-5}&This chart displays information of women's representation in different sectors. Law firms have the highest number of women employed with 60\% and tech has the lowest with 20\% of women. 
 & This chart shows women's representation in different sectors. We can see that Law firms have the highest number of women , while Tech has the lowest number of women.
\newline \textbf{BLEU Score: }42.3259\\ \cline{2-5}&This graph presents the proportion of women working in different  sectors. Law firms have the highest proportion of women , followed by financial groups and insurance. Tech has the lowest proportion of women represented - 20\%. 
 & The chart shows the proportion of women working in different  sectors. Law firms have the highest number of women , followed by financial groups. Insurance has the lowest number of women , and Tech has the highest number of women.
\newline \textbf{BLEU Score: }48.4305\\ \cline{2-5}\hline\multirow{2}{*}{Chart 1}&The UK has the highest gender pay gap which is more than double the lowest of Germany 
 & The chart shows that the UK has the highest gender pay gap of more than double that that that Germany.
\newline \textbf{BLEU Score: }38.2767\\ \cline{2-5}&This chart displays the gender pay gap in 3 countries. The UK has the highest percentage of a gap with 16\% , followed by Spain at 10\% and lastly Germany with 5\%. 
 & this chart shows the gender pay gap of 3 countries. The UK has the highest percentage gap with 16\% , followed by Spain at 10\% and lastly Germany at 5\%
\newline \textbf{BLEU Score: }64.3689\\ \cline{2-5}&This bar chart shows the gender pay gap in three countries , ; Germany, Span and the UK. It is displayed as a percentage , and the UK has a higher percentage. However it is not clear what the percentage is of, or which gender is favoured in the pay gap from the graph labels itself. 
 & This chart depicts the gender pay gap in three different countries : Germany, Span and the UK. The chart shows what the percentage is of, or which gender is favoured in the pay gap from the graph labels itself.
\newline \textbf{BLEU Score: }45.4068\\ \cline{2-5}\hline\multirow{2}{*}{Chart 2}&Young people are more likely to watch TV and go out rather than read a book 
 & Young people spend more time watching TV than go out. Read a book.
\newline \textbf{BLEU Score: }0.0\\ \cline{2-5}&This graph shows the activities young people undertake in the evenings. Most young people spend their evenings watching TV this accounts for 40\%. The next most common activity is going out with 30\% , then spending time with family at 20\% and lastly only 10\% of young people read books in the evening. The lack of book reading is probably due to more technology being ready accessible 
 & the table shows that 40\% of the next top 100 are in English , 30\% in French and 20\% in young people. The lack of book reading is probably due to more technology being ready accessible.
\newline \textbf{BLEU Score: }16.522\\ \cline{2-5}&This bar chart shows how young people spend their evenings. 40\% of young people spend their evenings watching TV (the most common activity) , while the least common activity is reading book (~ 10\%). it is unclear if young people could say that they do more than one of these things. 
 & This graph shows how young people spend their evenings. 40\% of them spend their evenings. Wat TV is the most popular activity for young people , while the least popular activity is reading book. 10\% of them do it is unclear if young people could say that they do more than one of these things.
\newline \textbf{BLEU Score: }53.1051\\ \cline{2-5}\hline\multirow{2}{*}{Chart 3}&The average salary for women has grown since 2000 by at least 10\% 
 & The chart shows the average salary for women in grown since 2000. The chart shows that the average salary for women has grown since 2000. The chart shows that the average salary for women has grown since 2000. The chart shows that the average salary for women has grown since 2000. The chart shows that the average salary for women has grown since 2000. The chart shows that the average salary for women has grown since 2000. The chart shows
\newline \textbf{BLEU Score: }7.0916\\ \cline{2-5}&This chart shows the median salary of women per year from the years 2000 - 2015. Over the years the median salary has increased it was 40,000 in 2000 and is now 56,000 in 2015. 
 & This graph shows the median salary of women per year. From the years 2000 - 2015. The chart shows that the median salary of women per year was increased by 40,000 in 2000 , to 56,000 in 2015
\newline \textbf{BLEU Score: }36.8664\\ \cline{2-5}&This graph presents the median salary of women from 2000 to 2015. Women ' s earnings have been steadily increasing over this time. 
 & The median salary of women has been steadily increasing. In 2000 it was 8\% and in 2015 it was 10\%.
\newline \textbf{BLEU Score: }12.7505\\ \cline{2-5}\hline\multirow{2}{*}{Chart 4}&The higher level of education the higher the salary 
 & This graph shows a trend that the higher the education status they have , the higher the salary they have
\newline \textbf{BLEU Score: }13.2648\\ \cline{2-5}&This chart displays the median salary per year of software engineers depending on their degree. Those with a PhD earn the most with 64,000 , then Masters at 54,000 , then Bachelor at 44,000 and those in the field with no degree earn 34,000. This corresponds with what you would expect that those with a better education and more knowledge earn more 
 & this chart shows median salary per year of software engineers depending on their degree. Those with a PhD earn the most at 64,000. Masters is 54,000 on Bachelor , 44,000 on no degree and 34,000 on a degree. This corresponds with what you would expect that those with a better education and more knowledge earn more.
\newline \textbf{BLEU Score: }56.4646\\ \cline{2-5}&This graph plots the median salary for software engineers split by their level of qualifications. PhD holders earn the highest median salary , followed by Masters holders , then Bachelors. Non-degree holders earn about half what PhD holders do. 
 & This graph shows the median salary for software engineers split by their level of qualifications. Persons holding PhD are the highest salary for software engineers with a PhD are then Bachelors and then Non-degree. Persons holding PhD are exactly half of those holding those with a PhD.
\newline \textbf{BLEU Score: }27.5246\\ \cline{2-5}\hline\multirow{2}{*}{Chart 5}&In Asia they ' re more likely to spend money on higher education compared to other continents 
 & Asia spends more money on higher education than other continents.
\newline \textbf{BLEU Score: }15.0037\\ \cline{2-5}&This chart shows the amount of dollars (millions) that was spent on higher education in different continents. Asia spent the most with 320 million , followed by North America with 200 million , Europe with 100 million spent and lastly Africa with 50 million. You expect Africa to be last due to poverty and lack of access to education. 
 & This chart shows how spent on higher education in continents. Asia spent the most at 320 million , followed by North America at 200 million. Europe was 100 million and lastly Africa was 50 million. You expect Africa to be last due to poverty and lack of access to education.
\newline \textbf{BLEU Score: }54.065\\ \cline{2-5}&This graph presents money spent on higher education in the year 2010 by different continents. It might be total money spent, it is not clear. Asia seems to spend the most , compared to Africa which spends the least. 
 & This chart shows the money spent on higher education in the year 2010 by different continents. It might be total money spent, it is not clear. Asia spent the most compared to Africa by far.
\newline \textbf{BLEU Score: }63.8546\\ \cline{2-5}\hline\multirow{2}{*}{Chart 6}&America has the highest number of top universities followed by Europe 
 & The chart shows that America has the highest number of top universities in each continent. We can see that Europe is the most popular , followed by America. Asia is the least popular , followed by Asia. Africa and Asia are the other two popular in each continent.
\newline \textbf{BLEU Score: }15.4196\\ \cline{2-5}&This chart shows the number of top 100 universities in each continent. America leads the way with 50 universities , then Europe with 30 , Asia with 20 and both Africa and Australia with 5 
 & This graph shows the number of top 100 universities in each continent. America leads the list with 50 , while Europe and Asia have 5 each. Africa and Australia show has having 5 each.
\newline \textbf{BLEU Score: }43.0994\\ \cline{2-5}&This graph displays the number of top 100 universities in each continent. The data presumably comes from a ranking source but this information is not provided. America has half the top universities , while Africa only has 4. 
 & This graph shows the number of top 100 universities in each continent. America has exactly half of the top universities while Africa has only 4.
\newline \textbf{BLEU Score: }27.4153\\ \cline{2-5}\hline\multirow{2}{*}{Chart 7}&Fast food is the highest factors in causing obesity 
 & Fast food is the highest cause of obesity
\newline \textbf{BLEU Score: }47.7503\\ \cline{2-5}&This chart displays causes of obesity. The highest percentage of obesity cases would be because of fast food at 60\% , followed by genetic reasons at 30\% and then lack of exercise at 10\%. 
 & The chart shows causes of obesity. The highest percentage is obesity with 60\% , followed by genetic factors at 30\% and lack of exercise at 10\%.
\newline \textbf{BLEU Score: }37.4851\\ \cline{2-5}&This graph shows three potential causes of obesity , and breaks down their relative contribution. Fast food emerges as the most important , followed by genetic factors. Lack of exercise is comparatively unimportant to these other two factors. 
 & This chart shows the three causes of obesity. The chart shows that fast food is the most popular , followed by genetic factors , and is least popular. It shows that Lack of exercise.
\newline \textbf{BLEU Score: }10.7504\\ \cline{2-5}\hline\multirow{2}{*}{Chart 8}&Students decided to study stem with more than double the other subjects 
 & Students chose to study stem in different subjects.
\newline \textbf{BLEU Score: }0.0\\ \cline{2-5}&This chart shows what subjects students choose to study. Most students pick STEM at approx 50\% , Literature and Medicine both account for 20\% and Philosophy is last with 10\% of students choosing this. 
 & The vast majority of students choose to study STEM - 50\% of all students. Literature and Medicine both have 20\% and Philosophy has the last 10\%. Students have been notified by email that their decision is being appealed.
\newline \textbf{BLEU Score: }12.8298\\ \cline{2-5}&This bar chart displays the percentage of students in different types of department. The modal category is STEM. It is unclear where other departments like History or Social Science fit. Philosophy is the smallest category - but this in unsurprising as it is one subject compared to the conglomeration of STEM subjects. 
 & This chart shows the percentage of students in different types of department. STEM subjects are the smallest and this in unsurprising as it is one subject compared to the conglomeration of STEM subjects.
\newline \textbf{BLEU Score: }39.1933\\ \cline{2-5}\hline\multirow{2}{*}{Chart 9}&Women are more likely to work in the arts and literature in university departments compared to mathematics and computer science 
 & Women are more represented in the arts and literature than more other arts and literature. They are more represented in the mathematics and computer science.
\newline \textbf{BLEU Score: }23.7133\\ \cline{2-5}&This shows the representation of women in various university departments. Most women are in the Literature department with 70\% being female , there are also a high number of women in Arts at 60\%. Mathematics is the lowest represented department for women only making up 10\% of staff. 
 & Most women are represented in various university departments. Most are represented in Literature department at 70\%. She is most represented in Arts department at 60\%. She is also least represented in Mathematics department at 10\%.
\newline \textbf{BLEU Score: }12.6905\\ \cline{2-5}&This bar chart displays the proportion of women in different university departments. Arts and Literature have higher representation , compared to Mathematics, Engineering and Computer Science.  & This chart shows the proportion of women in different university departments. We can see that both the Arts and Literature have much higher representation than Mathematics, Engineering and Computer Science.
\newline \textbf{BLEU Score: }45.3094\\ \cline{2-5}\hline \end{longtable}
    \end{landscape}
    \end{document}